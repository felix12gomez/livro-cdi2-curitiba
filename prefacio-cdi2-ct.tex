\newpage
\vspace*{9.0cm}
\begin{flushright}
\it Se a gente n\~ao pensar em ter mais,  fatalmente se ter\'a menos.
\end{flushright}

\newpage
\vspace{4cm}
\chapter*{Agradecimentos}
\addcontentsline{toc}{chapter}{Agradecimentos}
%------------------------------------------------

Agradeço de maneira muito especial às muitas pessoas que ajudaram de de diversas maneiras, generosamente, na criação desta monografia.

Tamb\'em fa\c co extensivo meus agradecimentos para o Prof. Jaime E. Muñoz Rivera por suas constantes observa\c c\~oes e sugest\~oes na forma\c c\~ao deste material. Sou grato a muitos professores do Departamento de Matemática-UFSC, responsaveis por diversas correções e melhoramentos.

Na UFSC, merecem especial destaque os meus alunos de Cálculo, pelas sugestões e reações através dos anos que contribuíram muito para aprofundar meus conhecimentos e minhas ideias de como apresentar o assunto e seu apoio constante nos momentos dif\'\i ceis.

Agrade\c co ao LNCC/MCTI pelas facilidades bibliográficas e computacionais, Plataformas SUN e IBM, sem as quais n\~ao houve-se podido entrar no mundo maravilhoso da computa\c c\~ao. Da mesma maneira ao IMPA/MCT que prestou-me as facilidades de biblio\-grafia e recursos computacionais, SUN e PC's sob a plataforma Windows  com ferramentas de excelente qualidade.

Sou imensamente grato a minha família, por levarem vidas \'\i ntegras e dedicadas, e por facilitarem as muitas viagens e compromissos fora de casa que precisei assumir.

Tamb\'em s\~ao necess\'arias algumas linhas de reconhecimento aos artistas an\^onimos da editora\c c\~ao eletr\'onica, pelo profundo envolvimento com o material e por suas capacidades, sensibilidades e cuidados ao lidar com a beleza do presente trabalho.

\begin{flushright}
\it F\'elix Pedro Quispe G\'omez\\
Fpolis, \today
\end{flushright}

\newpage

\chapter*{Prefácio}
\addcontentsline{toc}{chapter}{Prefácio}
%
O texto foi escrito do ponto de vista da matemático aplicad0, cujo interesse em cálculo avançado pode ser altamente teórico, intensamente prático ou algo no meio. Procuramos combinar uma exposição correta e precisa das teorias elementares. Tratamos dos principais métodos matemáticos da física e das engenharias. Neste sentido não é uma coleção de de ``receitas'' que se aplicam mas ou menos de memória. Os entes que nele comparecem são entes matemáticos definidos cuidadosamente. Mostramos sua propriedades elementares e os exemplos, emprestados da física e engenharia, colocam em destaque como devem ser utilizados os mesmos.

Escrevemos o presente trabalho, principalmente para o aluno da graduação em matemática, ciência ou engenharia, o qual, faz uma disciplina de cálculo avançado durante o segundo ano de estudo. O principal pré-requisito para se ler este texto é saber trabalhar com cálculo, o que pode ser obtido através de uma sequência de três semestres ou equivalente.

Este trabalho elemental e conciso não possui a ambição de ser um verdadeiro tratado; nas questões com tratamento extenso somente se enunciam os resultados essenciais, sem demostração.

A presentação do material é rigorosa e explora os métodos práticos para encontrar soluções. As vezes apelamos a métodos heurísticos nos casos que exijam intuição geométrica. O conteúdo é clássico e foi adaptado, sintetizado e estendido de excelentes trabalhos de Churchill \cite{chur}, Courant e Hilbert \cite{cou_hil}, Fritz John \cite{joh}

 Nos capítulos apresentados desenvolveu-se esforço especial para apresentar os assuntos da maneira mais clara e mais rigorosa possível; isto também se aplica a escolha das notações. Em cada capítulo, o nível aumenta gradualmente, evitando-se saltos e acúmulos de considerações  teóricas complexas.

Ao final de cada seção propomos uma coleção de exercícios. Estos estão, na medida do possível, classificados por ordem de dificuldade crescente. Alguns são de aplicação direta dos tópicos abordados; outros abordam questões novas, porém todos eles a nível dos conhecimentos correspondentes ao segundo ano de cálculo.


Os conhecimentos para ler este texto com proveito são os exigidos nas disciplinas de cálculo como algumas noções de álgebra linear e a teoria de variável complexa.

Concluímos, antecipando nosso agradecimento a nossos leitores pelo credito e confiança que possam brindar a presente obra e desejamos que nos façam chegar sugestões e ou críticas construtivas para poder corrigir os erros cometidos. Nesse sentido o autor assume a responsabilidade total dos mesmos.
%----------------------------------------------------------------------------------
